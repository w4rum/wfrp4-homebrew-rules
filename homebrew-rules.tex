\documentclass[parskip=full,11pt]{wfrp-short}
\usepackage[utf8]{inputenc}
\usepackage[T1]{fontenc}
\usepackage[useregional]{datetime2}
\usepackage{bold-extra}
\usepackage{xcolor}
\usepackage{graphicx}
\usepackage{csquotes}
\usepackage{amsmath} % for $\text{}$
\usepackage{enumitem}
\usepackage{hyperref}
\usepackage{background}
\usepackage{tabu}
\usepackage{float}
\usepackage{stfloats}
\usepackage{tcolorbox}
\usepackage[font={sc, sf, LARGE}, labelformat=empty]{caption}
\setlist{nosep}

\titlespacing{\section}{0pt}{10pt}{5pt}

\newcommand\urlpart[2]{$\underbrace{\text{\texttt{#1}}}{\text{#2}}$}
\raggedbottom

\hypersetup{
	pdftitle={WFRP4: Homebrew Rules of "Sacred Knights of Virginity"},
}

\title{WFRP4: Homebrew Rules}
%\subtitle{Rules and Interpretations}
\author{Sacred Knights of Virginity}

\backgroundsetup{
scale=1,
color=black,
opacity=.75,
angle=0,
contents={%
    \includegraphics[width=\paperwidth,height=\paperheight]{background.jpg}
  }%
}

\begin{document}
\maketitle
\thispagestyle{empty} % removed page number from title

\section*{} % No section number on intro, also removes it from TOC
\vspace{5pt}
This is a set of homebrew rules and interpretations of ambiguous statements
from the official material for Warhammer Fantasy Roleplay 4th Edition.

This document's structure is aimed at preserving the structure of the Core
Rulebook (CRB) and thus roughly reflects the position of the mentioned
mechanics in the CRB.

While this document is obviously based on work from Games Workshop and Cubile
7, it is merely meant to provide clarity to our local group and other fans.

Please don't sue us.

\pagebreak
\tableofcontents

%%%%%%%%%%%%%%
\pagebreak

%%%%%%%%%%%%%%%%%%%%%%%%%%%%%%%%%%%%%%%%%%%%%%%%%%%%%%%%%%%%%%%%%%%%%%%%%%%%%%%
\section{Rules}
%%%%%%%%%%%%%%%%%%%%%%%%%%%%%%%%%%%%%%%%%%%%%%%%%%%%%%%%%%%%%%%%%%%%%%%%%%%%%%%

\subsection{Bleeding Condition}
In addition to Heal Tests, spells, and prayers, you may make a
\textbf{Challenging Dexterity Test (+0)} as your Action to remove one
\textit{Bleeding} Condition if you use a bandage.
The bandage is only consumed on success, as failure is noticed immediately,
allowing you to try reapplying the same bandage next round.

\subsection{Stunned Condition}
When attempting to remove a \textit{Stunned} condition with an
\textbf{Endurance} Test, ignore all penalties inflicted by the \textit{Stunned}
conditions themselves.

E.g. a character with Endurance on 40 and 3 active \textit{Stunned} Conditions
would still roll on \textbf{Endurance (40)}.
Modifiers from all other sources still apply.

\subsection{Critical Wounds}
When taking a Critical Wound due to losing more Wounds than available, the
difference is treated as \enquote{excess damage}.

When determining the severity of the Critical Wound, add this modifier after
rolling 1d100:
\begin{itemize}
    \item If $\text{excess damage} < \text{Toughness Bonus}$: -20
    \item If $\text{excess damage} \in [\text{TB}, 2*\text{TB}]$: +0
    \item If $\text{excess damage} > 2* \text{Toughness Bonus}$: +20
\end{itemize}

The first case is included in the Core Rulebook.
It's mentioning here does not \enquote{stack} with the CRB.

After sustaining a Critical Wound in this manner, the character's wounds are
set to 0, applying the \textit{Prone} Condition as mentioned in the CRB.

%%%%%%%%%%%%%%%%%%%%%%%%%%%%%%%%%%%%%%%%%%%%%%%%%%%%%%%%%%%%%%%%%%%%%%%%%%%%%%%
\section{The\\Consumer's Guide}
%%%%%%%%%%%%%%%%%%%%%%%%%%%%%%%%%%%%%%%%%%%%%%%%%%%%%%%%%%%%%%%%%%%%%%%%%%%%%%%

All items here are either additions or modifications of already existing ones.
Whenever there is a name conflict, the homebrew rules take precedent.

\begin{figure*}[b]
    \begin{tcolorbox}[standard jigsaw, colback=yellow!5!orange, opacityback=.25, boxrule=0pt]
    \caption{Armour}
    \centering
    \begin{tabu}{c|c|c|c|c|c|c|c}
        \rowfont{\bfseries} Armour & Price & Enc & Availability &
            Penalty & Locations & APs & Qualities and Flaws\\
        \hline
        Gambeson & 12/- & 1 & Common & - & Arms, Body & 1 & -\\
    \end{tabu}
    \end{tcolorbox}
\end{figure*}



\end{document}
